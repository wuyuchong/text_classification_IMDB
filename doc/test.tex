% Options for packages loaded elsewhere
\PassOptionsToPackage{unicode}{hyperref}
\PassOptionsToPackage{hyphens}{url}
%
\documentclass[
  hyperref,]{ctexart}
\usepackage{lmodern}
\usepackage{amssymb,amsmath}
\usepackage{ifxetex,ifluatex}
\ifnum 0\ifxetex 1\fi\ifluatex 1\fi=0 % if pdftex
  \usepackage[T1]{fontenc}
  \usepackage[utf8]{inputenc}
  \usepackage{textcomp} % provide euro and other symbols
\else % if luatex or xetex
  \usepackage{unicode-math}
  \defaultfontfeatures{Scale=MatchLowercase}
  \defaultfontfeatures[\rmfamily]{Ligatures=TeX,Scale=1}
\fi
% Use upquote if available, for straight quotes in verbatim environments
\IfFileExists{upquote.sty}{\usepackage{upquote}}{}
\IfFileExists{microtype.sty}{% use microtype if available
  \usepackage[]{microtype}
  \UseMicrotypeSet[protrusion]{basicmath} % disable protrusion for tt fonts
}{}
\makeatletter
\@ifundefined{KOMAClassName}{% if non-KOMA class
  \IfFileExists{parskip.sty}{%
    \usepackage{parskip}
  }{% else
    \setlength{\parindent}{0pt}
    \setlength{\parskip}{6pt plus 2pt minus 1pt}}
}{% if KOMA class
  \KOMAoptions{parskip=half}}
\makeatother
\usepackage{xcolor}
\IfFileExists{xurl.sty}{\usepackage{xurl}}{} % add URL line breaks if available
\IfFileExists{bookmark.sty}{\usepackage{bookmark}}{\usepackage{hyperref}}
\hypersetup{
  hidelinks,
  pdfcreator={LaTeX via pandoc}}
\urlstyle{same} % disable monospaced font for URLs
\usepackage[margin=1in]{geometry}
\usepackage{longtable,booktabs}
% Correct order of tables after \paragraph or \subparagraph
\usepackage{etoolbox}
\makeatletter
\patchcmd\longtable{\par}{\if@noskipsec\mbox{}\fi\par}{}{}
\makeatother
% Allow footnotes in longtable head/foot
\IfFileExists{footnotehyper.sty}{\usepackage{footnotehyper}}{\usepackage{footnote}}
\makesavenoteenv{longtable}
\setlength{\emergencystretch}{3em} % prevent overfull lines
\providecommand{\tightlist}{%
  \setlength{\itemsep}{0pt}\setlength{\parskip}{0pt}}
\setcounter{secnumdepth}{5}
\usepackage{graphicx}
\usepackage{float}
\usepackage{indentfirst}
\setlength{\parindent}{4em}

\author{}
\date{}

\setCJKmainfont[BoldFont=SimHei,ItalicFont=STKaiti]{SimSun}

\begin{document}
\begin{titlepage}

\newcommand{\HRule}{\rule{\linewidth}{0.5mm}} % Defines a new command for the horizontal lines, change thickness here

\center % Center everything on the page
 
%----------------------------------------------------------------------------------------
%	HEADING SECTIONS
%----------------------------------------------------------------------------------------

\textsc{\LARGE Central University of Finance and Economics}\\[1.5cm] % Name of your university/college
\includegraphics[scale=1]{cufe.jpg}\\[1cm] % Include a department/university logo - this will require the graphicx package
\textsc{\Large 中央财经大学}\\[0.5cm] % Major heading such as course name
\textsc{\Large 大数据分析计算机基础课程}\\[0.5cm] % Minor heading such as course title

%----------------------------------------------------------------------------------------
%	TITLE SECTION
%----------------------------------------------------------------------------------------

\HRule \\[0.4cm]
{ \huge \bfseries 电影评论情感分析}\\[0.4cm] % Title of your document
\HRule \\[1.5cm]
 
%----------------------------------------------------------------------------------------
%	AUTHOR SECTION
%----------------------------------------------------------------------------------------

\begin{minipage}{0.4\textwidth}
\begin{center} \large
\textsc{\large 吴宇翀}\\[0.5cm] % Minor heading such as course title
\textsc{\large 谢一鸣}\\[0.5cm] % Minor heading such as course title
\textsc{\large 吴舫}\\[0.5cm] % Minor heading such as course title
\textsc{\large email@wuyuchong.com}\\[0.5cm] % Minor heading such as course title
\textsc{\large 指导老师:王成章}\\[0.5cm] % Minor heading such as course title
\end{center}

\end{minipage}\\[2cm]

% If you don't want a supervisor, uncomment the two lines below and remove the section above
%\Large \emph{Author:}\\
%John \textsc{Smith}\\[3cm] % Your name

%----------------------------------------------------------------------------------------
%	DATE SECTION
%----------------------------------------------------------------------------------------

{\large 2021年11月8日}\\[2cm] % Date, change the \today to a set date if you want to be precise

\vfill % Fill the rest of the page with whitespace

\end{titlepage}
\tableofcontents

\newpage

\hypertarget{ux6458ux8981}{%
\section{摘要}\label{ux6458ux8981}}

我们使用 IMDB
数据集进行文本分类。我们使用主题模型进行文本挖掘,之后进行情感分类。在文本预处理阶段,我们尝试使用词编码和词向量的方式,在训练阶段,我们构建了
DNN、LSTM、BERT 等多个深度学习模型进行训练,并进行了模型比较,最终达到了
90\% 的准确率。最后,为了进一步实现在超大文本集上进行训练,我们使用基于
Spark 的分布式算法在集群服务器上进行训练测试。\footnote{分布式模型在该小型数据集上没有优势,进行此项的意义在于对大型文本数据集可拓展性的技术储备,仅有在文本量级超过单机可承载上限时,分布式计算才具备意义}

\begin{longtable}[]{@{}lllll@{}}
\toprule
模型 & 计算配置 & 用时 & 准确率 & 可拓展性 \\
\midrule
\endhead
tokenize + DNN & 阿里云服务器 Xeon 8 核 CPU 32G 内存 & 10 分钟 & 60\% &
低-单机 \\
Word2Vec + LSTM & 阿里云服务器 Xeon 8 核 CPU 32G 内存 & 2 小时 & 80\% &
低-单机 \\
bert - 小型 & 阿里云服务器 Xeon 8 核 CPU 32G 内存 & 1 小时 & 90\% &
低-单机 \\
bert - AL & 阿里云服务器 Xeon 8 核 CPU 32G 内存 & 1.5小时 & 90\% &
低-单机 \\
bert - 标准 & 阿里云服务器 Xeon 8 核 CPU 32G 内存 & 3 小时 & 92\% &
低-单机 \\
spark & 中央财经大学大数据高性能分布式集群 & -- 分钟 & --\% & 高-集群 \\
\bottomrule
\end{longtable}

\end{document}
